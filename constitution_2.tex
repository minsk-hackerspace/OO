\documentclass[a4paper,fontsize=14pt,titlepage]{scrartcl}

\usepackage[T2A]{fontenc}
\usepackage[utf8]{inputenc}
\usepackage[russian]{babel}
\usepackage{indentfirst}
\usepackage{graphicx}
\usepackage[left=2cm,top=2cm,right=1.7cm,bottom=2cm,nohead,nofoot]{geometry}
\usepackage[font={small}]{caption}
\usepackage{wrapfig}
\usepackage{enumitem}
\usepackage{hyperref}


\setlist{nolistsep}
%\setlist[1]{\labelindent=\parindent}
%\setlist[itemize]{leftmargin=*}
\setlist[itemize,1]{label=---}

\setkomafont{section}{\normalfont\bfseries\centering}
\setkomafont{paragraph}{\normalfont}
\setkomafont{subparagraph}{\normalfont}
\setkomafont{title}{\rmfamily \huge}

\newcommand{\longpage}{\enlargethispage{\baselineskip}}
\newcommand{\shortpage}{\enlargethispage{-\baselineskip}}

% deep magic follows, don't touch it!
\makeatletter
\@newctr{paragraph}[section]
\@newctr{subparagraph}[paragraph]
\newenvironment{numberedpars}{%
  \addtocounter{secnumdepth}{1}
  \renewcommand\theparagraph{\arabic{section}.\arabic{paragraph}.}
  \renewcommand\@seccntformat[1]
  {\expandafter\ifx\csname##1\endcsname\paragraph\csname
  the##1\endcsname\else\csname the##1\endcsname\fi}
  \def\paragraph{\par\addvspace{3.25ex \@plus1ex \@minus.2ex}{\raggedsection\normalfont\sectfont\nobreak\size@paragraph\refstepcounter{paragraph}\@seccntformat{paragraph}}\,}
  \let\old@par=\par
  \def\new@par{\let\par=\old@par\paragraph{}\let\par=\new@par}
  \let\par=\new@par
  \par
}{
  \addtocounter{secnumdepth}{-1}
}
\newenvironment{numberedsubpars}{%
  \addtocounter{secnumdepth}{1}
  \renewcommand\thesubparagraph{\arabic{section}.\arabic{paragraph}.\arabic{subparagraph}.}
  \renewcommand\@seccntformat[1]
  {\expandafter\ifx\csname##1\endcsname\subparagraph\csname
  the##1\endcsname\else\csname the##1\endcsname\fi}
  \def\subparagraph{\par\addvspace{3.25ex \@plus1ex \@minus .2ex}{\raggedsection\normalfont\sectfont\nobreak\size@paragraph\refstepcounter{subparagraph}\@seccntformat{subparagraph}}\,}
  \let\old@@par=\par
  \def\new@@par{\let\par=\old@par\subparagraph{}\let\par=\new@@par}
  \let\par=\new@@par
  \par
}{
  \addtocounter{secnumdepth}{-1}
  \let\par=\old@@par
}

\def\emph{\textbf}

\renewcommand\thesection{\arabic{section}.}
\let\@@@section=\section
% \renewcommand\section[1]{\@@@section{\uppercase{#1}}}
% end of magic

\makeatother

\sloppy

\newcommand{\OOname}{«Открытая лаборатория технического творчества»}
\newcommand{\OONAME}{«ОТКРЫТАЯ ЛАБОРАТОРИЯ ТЕХНИЧЕСКОГО ТВОРЧЕСТВА»}
\newcommand{\OOnameb}{«Адкрытая лабараторыя тэхнічнай творчасці»}
\newcommand{\OO}{ОО \OOname}

\begin{document}

\setkomafont{title}{\bfseries}

\date{\large г. Минск\\2013}

\titlehead{\raggedleft \begin{minipage}{15em}%
Принят на\\%
Учредительном Cобрании\\%
общественного объединения\\%
«Открытая лаборатория\\%
технического творчества»\\%
29 июля 2013 года%%
\end{minipage}}


\title{\Large%
УСТАВ\\%
\textbf{\large ОБЩЕСТВЕННОГО ОБЪЕДИНЕНИЯ}\\%
«Открытая лаборатория технического творчества»\\[1ex]%
(ОО «Открытая лаборатория\\%
технического творчества»)\\[4ex]%
СТАТУТ\\%
\textbf{\large ГРАМАДСКАГА АБ’ЯДНАННЯ}\\%
«Адкрытая лабараторыя тэхнічнай творчасці»\\[1ex]%
(ГА «Адкрытая лабараторыя\\%
тэхнічнай творчасці»)%
}
\maketitle


\section{ОБЩИЕ ПОЛОЖЕНИЯ}


\begin{numberedpars}
Общественное объединение \OOname\
(далее "--- \OO)
является добровольным объединением граждан,
объединившихся на основе общности интересов в установленном законодательством порядке.
\end{numberedpars}
\begin{numberedpars}
Полное название объединения:
\end{numberedpars}
\begin{itemize}
\item на русском языке: \emph{Общественное объединение \OOname};
\item на белорусском языке: \emph{Грамадскае аб’яднанне \OOnameb}.
\end{itemize}
Сокращенное название объединения:
\begin{itemize}
\item на русском языке: \emph{\OO};
\item на белорусском языке: \emph{ГА \OOnameb}.
\end{itemize}
\begin{numberedpars}
\OO\ имеет статус местного общественного объединения.
Территория распространения деятельности \OO\ "--- г. Минск.
\end{numberedpars}
\begin{numberedpars}
\OO\ создается и действует в соответствии с законодательством Республики Беларусь
и настоящим Уставом.
\end{numberedpars}
\begin{numberedpars}
\OO\ с момента регистрации является юридическим лицом,
несет самостоятельную ответственность по своим обязательствам,
имеет обособленное имущество,
самостоятельный баланс и счета в банках,
от своего имени выступает во взаимоотношениях с юридическими и физическими лицами,
может быть истцом и ответчиком в судах,
имеет печать,
бланки со своим наименованием,
может иметь собственную символику,
которая подлежит регистрации в установленном порядке.
\end{numberedpars}
\begin{numberedpars}
\OO\ может вступать в международные общественные объединения,
созданные на территории иностранных государств,
союзы, участвовать в их создании,
поддерживать прямые международные контакты и связи в соответствии с законодательством.
\end{numberedpars}
\begin{numberedpars}
Юридический адрес \OO: г. Минск, пр. Независимости 58, корп. 6, помещения № 12.
\end{numberedpars}



\newpage\section{ЦЕЛЬ, ЗАДАЧИ, ПРЕДМЕТ И МЕТОДЫ ДЕЯТЕЛЬНОСТИ}


\begin{numberedpars}
Целями \OO\ являются развитие современных информационных
технологий, науки, техники и электроники, повышение уровня знаний и навыков населения в данной области.
\end{numberedpars}
\begin{numberedpars}
Задачами \OO\ являются:
\end{numberedpars}
\begin{itemize}
\item поддержка и популяризация технического творчества;
\item cоздание и обеспечение функционирования центров технического творчества;
\item создание благоприятной среды для обмена знаниями и опытом, общения и совместной работы для участников проектов,
связанных с техническим творчеством;
\item распространение культуры и принципов свободного программного и аппаратного обеспечения.
\end{itemize}
\begin{numberedpars}
Предметом деятельности \OO\ являются просветительская, образовательная деятельность.
\end{numberedpars}
\begin{numberedpars}
Для достижения своей цели и решения задач \OO\ в порядке,
установленном законодательством, использует следующие методы деятельности:
\end{numberedpars}
\begin{itemize}
\item распространение информации о деятельности сообщества;
\item проведение обучающих семинаров и лекций;
\item участие и организация выставок, конференций, круглых столов и иных аналогичных мероприятий, представляющих интерес для
членов технического сообщества;
\item проведение конкурсов и соревнований;
\item создание и обеспечение работы центров технического творчества, клубов радиолюбителей-конструкторов, программистов;
\item содействие и поддержка в реализации оригинальных проектов участников сообщества;
\item инициирование и участие в совместных проектах с учебными (государственными и негосударственными) учереждениями,
научно-образовательными центрами и им подобными;
\item создание и публикация экспертных исследований, аналитических обзоров,
информационных и обучающих материалов, рекомендаций, касающихся актуальных вопросов социальных,
экономических, правовых, технологических аспектов разработки и сопровождения свободного программного и аппаратного обеспечения,
открытых форматов данных;
\item в установленном законодательством порядке вступление в международные общественные (неправительственные) объединения и
союзы, установление и поддержка прямых международных контактов и связей, заключение для этих целей соответствующих
соглашений;
\item получение финансовой и иной поддержки из различных не запрещённых законодательством Республики Беларусь источников;
\item иные методы, не противоречащие законодательству Республики Беларусь.
\end{itemize}
Деятельность, на осуществление которой требуется специальное разрешение (лицензия),
осуществляется только после получения необходимого разрешения (лицензии) в установленном порядке.




\newpage\section{ЧЛЕНСТВО, ПРАВА И ОБЯЗАННОСТИ ЧЛЕНОВ\\ ОО \OONAME}


\begin{numberedpars}
Членами \OO\ могут быть, в соответствии с законодательством,
граждане Республики Беларусь, а также иностранные граждане и лица без гражданства,
достигшие возраста 18 лет, согласные с положениями Устава и желающие участвовать в достижении его целей.
Лица, в возрасте от 16 до 18 лет, могут вступать в \OO\ с письменного
согласия своих законных представителей.
\end{numberedpars}
\begin{numberedpars}
Прием в члены \OO\ осуществляется решением
Правления \OO\ по заявлению вступающего.
Прекращение членства может быть осуществлено путем выхода из членов объединения либо в случае исключения из числа членов
\OO.
\end{numberedpars}
\begin{numberedpars}
Размер вступительного и членских взносов, порядок их уплаты определяется Правлением \OO.
Правление \OO\ может принять решение об освобождении
отдельных граждан от уплаты вступительного и членских взносов, для отдельных членов
\OO\ размер вступительного и членских взносов может быть снижен.
\end{numberedpars}
\begin{numberedpars}
Выход из членов \OO\ осуществляется путем подачи заявления
в Правление \OO\, при этом членство считается прекращенным с даты, указанной в заявлении.
\end{numberedpars}
\begin{numberedpars}
Решение об исключении из числа членов \OO\ может быть принято
Правлением \OO.
Членство считается прекращенным с даты, указанной в решении Правления \OO\ об
исключении из числа членов \OO.
\end{numberedpars}

\begin{numberedpars}
Член \OO\ имеет право:
\end{numberedpars}
\begin{itemize}
\item принимать участие в мероприятиях \OO, в заседаниях высшего органа;
\item избирать и выдвигать свою кандидатуру на выборах в выборные органы \OO;
\item получать от органов и должностных лиц \OO\ информацию,
касающуюся деятельности \OO;
\item обжаловать решения органов и должностных лиц \OO\ в порядке,
предусмотренном Уставом и законодательством;
\item вносить предложения относительно деятельности \OO\
на рассмотрение выборных органов объединения;
\item вносить добровольные пожертвования для поддержки деятельности \OO;
\item свободного выхода из членов \OO.
\end{itemize}

\begin{numberedpars}
Член \OO\ обязан:
\end{numberedpars}
\begin{itemize}
\item выполнять требования Устава;
\item участвовать в работе по выполнению целей и задач \OO;
\item не совершать действий, наносящих материальный ущерб или причиняющих вред
деловой репутации \OO.
\end{itemize}
\begin{numberedpars}
Учет членов в \OO\ осуществляется Правлением \OO путем ведения списка членов,
который редактируется по мере необходимости и обновляется по мере вступления и выбытия членов.
\end{numberedpars}

\newpage\section{ВЫСШИЙ И ВЫБОРНЫЕ ОРГАНЫ\\ ОО \OONAME}

\begin{numberedpars}
\OO\ является цельным общественным объединением, не имеющим в своём
составе организационных структур.
\end{numberedpars}
\begin{numberedpars}
Высшим органом \OO\ является Общее Собрание \OO\ (далее "--- Общее Собрание),
которое созывается Правлением \OO\ по мере необходимости,
но не реже одного раза в год.\\
Общее Собрание может быть созвано по требованию Правления \OO\ либо Ревизора
\OO, либо по инициативе не менее одной пятой части всех членов \OO.\\
Время, место проведения, повестка дня Общего Собрания определяются Правлением \OO\
и сообщаются членам объединения не позднее, чем за 5 (пять) дней до дня Общего Собрания.\\
Общее Собрание считается правомочным, если в нем участвует не менее половины членов \OO.\\
Форма и порядок голосования устанавливается Собщим Собранием.
\end{numberedpars}
\begin{numberedpars}
К компетенции Общего Собрания относится:
\end{numberedpars}
\begin{itemize}
\item определение основных направлений и форм деятельности \OO;
\item утверждение названия общественного объединения;
\item принятие Устава \OO, внесение в него изменений и/или дополнений,
кроме случаев, когда в соответствии с законодательством и Уставом изменения и/или дополнения в Устав могут быть внесены Правлением;
\item избрание сроком на 1 (один) год Правления в составе не менее 5 (пяти) человек;
\item избрание из числа членов Правления сроком на 1 (один) год Председателя Правления;
\item избрание из числа членов Правления сроком на 1 (один) год Заместителя Председателя Правления;
\item избрание сроком на 1 (один) год Ревизора;
\item заслушивание отчетов Правления, Ревизора;
\item принятие решения о реорганизации или ликвидации \OO.
\end{itemize}
Общее Собрание может принять к рассмотрению любой другой вопрос деятельности
\OO.
\begin{numberedpars}
В период между Общими Собраниями деятельностью \OO\ руководит
Правление, являющееся руководящим органом \OO.
Правление избирается сроком на 1 (один) год.
Заседания Правления созываются Председателем Правления по мере необходимости, но не реже одного раза в 3 (три) месяца.
\end{numberedpars}
\begin{numberedpars}
Правление правомочно, если на его заседании присутствует не менее половины членов Правления.
\end{numberedpars}
\begin{numberedpars}
Правление \OO:
\end{numberedpars}
\begin{itemize}
\item организует деятельность \OO, исходя из его цели, задач и методов;
\item созывает и организует работу Общего Собрания;
\item утверждает отчеты о доходах и расходах \OO\ и сметы расходов на
будущий период;
\item принимает решения об участии в создании или вступлении в союзы и ассоциации;
\item принимает решения о создании и ликвидации юридических лиц, созданных при
участии \OO\ в соответствии с законодательством, утверждает их
уставы и руководителей;
\item вносит изменения и/или дополнения в Устав, связанные с переменой юридического адреса
либо обусловленные изменениями в законодательстве;
\item утверждает штатное расписание и должностные оклады, нанимает и увольняет
штатных работников;
\item принимает решения о приобретении, распоряжении и отчуждении имущества \OO;
\item заслушивает отчеты Председателя Правления, руководителей созданных \OO\ юридических лиц;
\item утверждает образцы печати, штампов, символики \OO;
\item решает иные вопросы деятельности \OO, не относящиеся согласно
Уставу к компетенции других органов \OO.
\end{itemize}
\begin{numberedpars}
Председатель Правления является руководителем юридического лица.
Председатель Правления входит в состав Правления по должности.
\end{numberedpars}
\begin{numberedpars}
Председатель Правления:
\end{numberedpars}
\begin{itemize}
\item созывает заседания Правления, определяет время и место их проведения,
оповещает всех членов Правления не позднее чем за 2 (два) дня до заседания;
\item председательствует на заседаниях Правления;
\item осуществляет контроль за выполнением решений Правления;
\item обеспечивает выполнение решений Общего Собрания и Правления;
\item без доверенности действует от имени \OO, представляет его интересы;
\item принимает решения по вопросам, которые не отнесены к компетенции Общего Собрания и Правления;
\item принимает решения и издает распоряжения по текущим вопросам деятельности \OO;
\item заключает договоры от имени \OO, выдает доверенности;
\item открывает расчетные и другие счета в банках;
\item распоряжается имуществом и средствами \OO\ в пределах, устанавливаемых Правлением;
\item решает другие вопросы, связанные с деятельностью \OO\
и не отнесенные к компетенции иных органов.
\end{itemize}
При отсутствии Председателя Правления все его обязанности исполняет Заместитель Председателя Правления.
\begin{numberedpars}
Для осуществления внутренней проверки финансово-хозяйственной деятельности,
а также внутреннего контроля за соответствием деятельности \OO\
учредительным документам и законодательству Общее Собрание избирает Ревизора, подотчётного Общему Собранию.
\end{numberedpars}
\begin{numberedpars}
Ревизор \OO:
\end{numberedpars}
\begin{itemize}
\item контролирует деятельность \OO;
\item проверяет бухгалтерские счета и книги, просматривает документы \OO\ в любое время
в период действия своих полномочий;
\item проверяет обоснованность ответов на письма, жалобы, предложения членов \OO;
\item проверяет организацию делопроизводства и отчетности \OO.
\end{itemize}
\begin{numberedpars}
Ревизор проводит проверки по мере необходимости, но не реже раза в 1 (один) год.
Ревизор в случае необходимости вправе привлекать к своей работе специалистов для консультаций и участия в проведении ревизий.
\end{numberedpars}
\begin{numberedpars}
Решения коллегиальных органов \OO\ оформляются протоколами.
Председатель Правления издает приказы и распоряжения.
Ревизор оформляет проверки актами и справками.
\end{numberedpars}
\begin{numberedpars}
Решения коллегиальных органов \OO\ принимаются открытым голосованием простым большинством голосов
от числа присутствующих, если Уставом не оговорено иное.
\end{numberedpars}
\begin{numberedpars}
Решение Председателя может быть обжаловано Ревизору.
Ревизор в течение месяца обязан рассмотреть жалобу и вынеси по ней решение.
Решение Ревизора может быть обжаловано на Общем Собрании.
Ревизор в месячный срок обязан принять решение по существу рассматриваемой жалобы.
Ответ компетентного органа направляется в пятидневный срок заявителю.
Общее Собрание может принять к рассмотрению любой вопрос по жалобе на неправомерные действия
должностных лиц и выборных органов объединения.
\end{numberedpars}


\newpage\section{ФИНАНСОВЫЕ СРЕДСТВА И ИМУЩЕСТВО}


\begin{numberedpars}
\OO\ может иметь в собственности любое имущество, необходимое
для материального обеспечения уставной деятельности, за исключением объектов,
которые, согласно закону, могут находиться только в собственности государства.
\end{numberedpars}
\begin{numberedpars}
Денежные средства и имущество \OO\ формируются из:
\end{numberedpars}
\begin{itemize}
\item вступительных и членских взносов;
\item добровольных пожертвований физических и юридических лиц;
\item поступлений от проводимых в уставных целях мероприятий в соответствии с законодательством;
\item отчислений от созданных \OO\ предприятий;
\item других поступлений, не запрещенных законодательством.
\end{itemize}
\begin{numberedpars}
Денежные средства \OO\ расходуются на:
\end{numberedpars}
\begin{itemize}
\item выполнение задач, стоящих перед \OO;
\item финансирование планов, программ, проектов исследований в соответствии с уставными целями и задачами \OO;
\item обеспечение работы выборных органов \OO;
\item выплату заработной платы штатным сотрудникам исполнительного аппарата и членам \OO,
их премирование и поощрение в соответствии с настоящим Уставом и законодательством;
\item создание юридических лиц, деятельность которых направлена на решение уставных задач;
\item развитие материально-технической базы \OO;
\item благотворительную деятельность;
\item на иные цели, предусмотренные настоящим Уставом, законодательством.
\end{itemize}
\begin{numberedpars}
Денежные cредства и имущество \OO\ не могут перераспределяться между его членами.
\end{numberedpars}
\begin{numberedpars}
\OO\ несет ответственность по принятым на себя обязательствам
всем принадлежащим ему имуществом.
\OO\ не отвечает по обязательствам своих членов,
а они не отвечают по его обязательствам.
\end{numberedpars}
\begin{numberedpars}
В случае прекращения членства в \OO\ финансовые средства и имущество,
переданные его членами \OO\ в собственность безвозмездно, возврату не подлежат.
Материальные средства, переданные \OO\
его членами во временное владение и пользование, возвращаются в соответствии с
условиями договоров, на основании которых это владение и пользование осуществлялось.
\end{numberedpars}


\newpage\section{ПРАВА ОО \OONAME}

\begin{numberedpars}
Для достижения уставных цели и задач \OO\ вправе:
\end{numberedpars}
\begin{itemize}
\item осуществлять деятельность, направленную на достижение уставной цели;
\item беспрепятственно получать и распространять информацию, имеющую отношение к своей деятельности,
в порядке, установленном законодательством, учреждать собственные и пользоваться в установленном порядке государственными
средствами массовой информации, осуществлять издательскую деятельность;
\item представлять и защищать права и законные интересы своих членов в государственных органах и иных организациях;
\item поддерживать связи с другими некоммерческими организациями;
\item создавать некоммерческие организации, вступать в союзы (ассоциации);
\item иметь иные права, предусмотренные законодательными актами.
\end{itemize}
\begin{numberedpars}
\OO\ может осуществлять в установленном порядке предпринимательскую деятельность лишь постольку,
поскольку она необходима для уставных целей, ради которых оно создано, соответствует этим целям и отвечает предмету деятельности.
Такая деятельность может осуществляться только посредством образования коммерческих организаций и (или) участия в них.
\end{numberedpars}


\newpage\section{ПРЕКРАЩЕНИЕ ДЕЯТЕЛЬНОСТИ\\ ОО \OONAME}


\begin{numberedpars}
Прекращение деятельности \OO\ происходит путем его реорганизации или ликвидации.
\end{numberedpars}
\begin{numberedpars}
Реорганизация \OO\ производится по решению Общего Собрания.
Ликвидация \OO\ производится по решению Общего Собрания, либо по решению суда.
Решение о ликвидации направляется в регистрирующий орган и публикуется в периодическом печатном издании,
определенном актами законодательства.
Ликвидация производится ликвидационной комиссией, созданной органом, принявшим решение о ликвидации.
\end{numberedpars}
\begin{numberedpars}
При ликвидации средства и имущество \OO,
оставшиеся после полного удовлетворения всех имущественных требований кредиторов,
используются на цели, предусмотренные Уставом, если денежные средства и иное имущество объединения,
в соответствии с законодательными актами, не подлежат обращению в доход государства.
\end{numberedpars}


\end{document}
