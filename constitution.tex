\documentclass[a4paper,fontsize=14pt,titlepage]{scrartcl}

\usepackage[T2A]{fontenc}
\usepackage[utf8]{inputenc}
\usepackage[russian]{babel}
\usepackage{indentfirst}
\usepackage{graphicx}
\usepackage[left=3cm,top=2cm,right=1.7cm,bottom=2cm,nohead,nofoot]{geometry}
\usepackage[font={small}]{caption}
\usepackage{wrapfig}
\usepackage{enumitem}
\usepackage{hyperref}


\setlist{nolistsep}
%\setlist[1]{\labelindent=\parindent}
%\setlist[itemize]{leftmargin=*}
\setlist[itemize,1]{label=---}

\setkomafont{section}{\normalfont\bfseries\centering}
\setkomafont{paragraph}{\normalfont}
\setkomafont{subparagraph}{\normalfont}
\setkomafont{title}{\rmfamily \huge}

\newcommand{\longpage}{\enlargethispage{\baselineskip}}
\newcommand{\shortpage}{\enlargethispage{-\baselineskip}}

% deep magic follows, don't touch it!
\makeatletter
\@newctr{paragraph}[section]
\@newctr{subparagraph}[paragraph]
\newenvironment{numberedpars}{%
  \addtocounter{secnumdepth}{1}
  \renewcommand\theparagraph{\arabic{section}.\arabic{paragraph}.}
  \renewcommand\@seccntformat[1]
  {\expandafter\ifx\csname##1\endcsname\paragraph\csname
  the##1\endcsname\else\csname the##1\endcsname\fi}
  \def\paragraph{\par\addvspace{3.25ex \@plus1ex \@minus.2ex}{\raggedsection\normalfont\sectfont\nobreak\size@paragraph\refstepcounter{paragraph}\@seccntformat{paragraph}}\,}
  \let\old@par=\par
  \def\new@par{\let\par=\old@par\paragraph{}\let\par=\new@par}
  \let\par=\new@par
  \par
}{
  \addtocounter{secnumdepth}{-1}
}
\newenvironment{numberedsubpars}{%
  \addtocounter{secnumdepth}{1}
  \renewcommand\thesubparagraph{\arabic{section}.\arabic{paragraph}.\arabic{subparagraph}.}
  \renewcommand\@seccntformat[1]
  {\expandafter\ifx\csname##1\endcsname\subparagraph\csname
  the##1\endcsname\else\csname the##1\endcsname\fi}
  \def\subparagraph{\par\addvspace{3.25ex \@plus1ex \@minus .2ex}{\raggedsection\normalfont\sectfont\nobreak\size@paragraph\refstepcounter{subparagraph}\@seccntformat{subparagraph}}\,}
  \let\old@@par=\par
  \def\new@@par{\let\par=\old@par\subparagraph{}\let\par=\new@@par}
  \let\par=\new@@par
  \par
}{
  \addtocounter{secnumdepth}{-1}
  \let\par=\old@@par
}

\def\emph{\textbf}

\renewcommand\thesection{\arabic{section}.}
\let\@@@section=\section
% \renewcommand\section[1]{\@@@section{\uppercase{#1}}}
% end of magic

\makeatother

\sloppy

\newcommand{\OOname}{«Открытая лаборатория технического творчества»}
\newcommand{\OONAME}{«ОТКРЫТАЯ ЛАБОРАТОРИЯ ТЕХНИЧЕСКОГО ТВОРЧЕСТВА»}
\newcommand{\OOnameb}{«Адкрытая лабараторыя тэхнічнай творчасці»}
\newcommand{\OO}{ПОО \OOname}

\newcommand{\Ustav}{Устав Просветительского общественного объединения «Открытая лаборатория технического творчества»}

\hypersetup{
    pdftitle={\Ustav},
    pdfauthor={Просветительское общественное объединение \OOname},
    pdfsubject={\Ustav},
    pdfkeywords={},
    bookmarksnumbered=true,
    bookmarksopen=true,
    bookmarksopenlevel=1,
    colorlinks=true,
    pdfstartview=Fit,
    pdfpagemode=UseOutlines,    % this is the option you were lookin for
    pdfpagelayout=TwoPageRight
}

\begin{document}

\setkomafont{title}{\bfseries}

\date{\large г. Минск\\2015}

\titlehead{\raggedleft \begin{minipage}{15em}%
Утвержден\\%
Учредительным Общим\\% 
Собранием Просветительского 
общественного объединения\\% 
«Открытая лаборатория\\%
технического творчества»\\%
года 2015 марта «9»%%
\end{minipage}}


\title{\Large%
УСТАВ\\%
\textbf{\large ПРОСВЕТИТЕЛЬСКОГО ОБЩЕСТВЕННОГО ОБЪЕДИНЕНИЯ}\\%
«Открытая лаборатория технического творчества»\\[1ex]%
(ПОО «Открытая лаборатория\\%
технического творчества»)\\[4ex]%
СТАТУТ\\%
\textbf{\large АСВЕТНИЦКАГА ГРАМАДСКАГА АБ’ЯДНАННЯ}\\%
«Адкрытая лабараторыя тэхнічнай творчасці»\\[1ex]%
(АГА «Адкрытая лабараторыя\\%
тэхнічнай творчасці»)%
}
\maketitle


\section{ОБЩИЕ ПОЛОЖЕНИЯ}


\begin{numberedpars}
Просветительское общественное объединение \OOname\
(далее по тексту настоящего Устава "--- \OO)
является добровольным объединением граждан,
в установленном законодательством порядке объединившихся 
на основе общности интересов для удовлетворения 
нематериальных потребностей и достижения уставных целей.
\end{numberedpars}
\begin{numberedpars}
Полное название объединения:
\end{numberedpars}
\begin{itemize}
\item на русском языке: \emph{Просветительское общественное объединение \OOname};
\item на белорусском языке: \emph{Асветницкае грамадскае аб’яднанне \OOnameb}.
\end{itemize}
Сокращенное название объединения:
\begin{itemize}
\item на русском языке: \emph{\OO};
\item на белорусском языке: \emph{АГА \OOnameb}.
\end{itemize}
\begin{numberedpars}
\OO\ имеет статус местного общественного объединения.
Территория распространения деятельности \OO\ "--- город Минск.
\end{numberedpars}
\begin{numberedpars}
\OO\ создается и действует в соответствии с законодательством Республики Беларусь
и на основании настоящего Устава.
\end{numberedpars}
\begin{numberedpars}
\OO\ с момента регистрации является юридическим лицом,
несет самостоятельную ответственность по своим обязательствам,
имеет обособленное имущество,
самостоятельный баланс и счета в банках,
от своего имени выступает во взаимоотношениях с юридическими и физическими лицами,
может быть истцом и ответчиком в судах,
имеет печать,
бланки со своим наименованием,
может иметь собственную символику,
которая подлежит регистрации в установленном порядке.
\end{numberedpars}
\begin{numberedpars}
\OO\ может вступать в международные общественные объединения,
созданные на территории иностранных государств,
участвовать в их создании,
поддерживать прямые международные контакты и связи в соответствии с законодательством Республики Беларусь,
вступать во взаимодействие с общественными объединениями и иными организациями, 
взаимодействие с которыми не противоречат законодательству Республики Беларусь.
\end{numberedpars}
\begin{numberedpars}
Юридический адрес \OO: г. Минск, ул. Леонида Беды 45, комн. 755.
\end{numberedpars}



\newpage\section{ЦЕЛИ, ЗАДАЧИ, ПРЕДМЕТ, МЕТОДЫ ДЕЯТЕЛЬНОСТИ И ПРАВА ПОО \OONAME}


\begin{numberedpars}
Целями \OO\ являются
содействие развитию технического творчества, 
изучение, популяризация и развитие достижений технологий, науки и техники, 
повышение уровня знаний и навыков населения в данной области.
\end{numberedpars}
\begin{numberedpars}
Задачами \OO\ являются:
\end{numberedpars}
\begin{itemize}
\item поддержка и популяризация технического творчества;
\item создание благоприятной среды для обмена знаниями и опытом, 
общения и совместной работы в области технического творчества;
\item содействие и поддержка в реализации проектов членов \OO;
\item распространение принципов открытости программного и аппаратного обеспечения.
\end{itemize}
\begin{numberedpars}
Предметом деятельности \OO\ является просветительская деятельность 
в области технического творчества, 
направленная на достижение уставных целей и решение уставных задач.
\end{numberedpars}
\begin{numberedpars}
Для достижения своих целей и решения своих задач \OO\ в порядке,
установленном законодательством, использует следующие методы деятельности:
\end{numberedpars}
\begin{itemize}
\item распространение информации о деятельности объединения и его членов;
\item проведение обучающих семинаров и лекций;
\item участие и организация выставок, конференций, круглых столов и иных аналогичных мероприятий, представляющих интерес для
членов объединения;
\item проведение конкурсов и соревнований в области технического творчества;
\item содействие созданию и поддержка работы центров технического творчества, 
клубов радиолюбителей-конструкторов, программистов,
создаваемых в соответствии с законодательством;
\item инициирование и участие в совместных проектах с учебными
(государственными и негосударственными) учреждениями,
научно-образовательными центрами и им подобными;
%% прелогают добавить такой пункт:
%% \item защита прав и законных интересов, а также представление 
%%законных интересов своих членов в государственных органах и иных организациях;
\item cоздание и публикация экспертных исследований, аналитических обзоров, 
информационных и обучающих материалов, рекомендаций, 
касающихся актуальных вопросов социальных, экономических, правовых, 
технологических аспектов разработки и сопровождения свободного программного и аппаратного обеспечения, 
открытых форматов данных.
%% убрать следующие пункты:
%% \item в установленном законодательством порядке вступление в местные и международные
%% общественные (неправительственные) объединения и союзы,
%% установление и поддержка прямых международных контактов и связей,
%% заключение для этих целей соответствующих соглашений;
%% \item получение финансовой и иной поддержки из различных не запрещенных законодательством Республики Беларусь источников;
%% \item иные методы, не противоречащие законодательству Республики Беларусь.
\end{itemize}
\begin{numberedpars}
Деятельность, на осуществление которой требуется специальное разрешение (лицензия), 
осуществляется \OO только после получения необходимого разрешения (лицензии) в установленном порядке.
\end{numberedpars}
\begin{numberedpars}
\OO имеет права, предусмотренные законодательными актами. 
\end{numberedpars}
\begin{numberedpars}
\OO может осуществлять в установленном порядке предпринимательскую деятельность лишь постольку, 
поскольку она необходима для уставных целей, ради которых оно создано, 
соответствует этим целям и отвечает предмету деятельности. 
Такая деятельность может осуществляться только посредством образования 
коммерческих организаций и (или) участия в них.
\end{numberedpars}

%% Вам (т.е. нам) надо поработать над задачами, вернее даже над связкой цели-задачи-методы. К вашей организации могут придраться в том контексте, что из устава не понятно, чем вы собираетесь занимать и для чего создаетесь (это достаточно распространенное явление). Например, стоит добавить что-то типа привлечение молодежи к занятию техническим творчеством, организация их досуга. 


\newpage\section{ЧЛЕНСТВО, ПРАВА И ОБЯЗАННОСТИ ЧЛЕНОВ ПОО \OONAME}


\begin{numberedpars}
Членами \OO\ могут быть, в соответствии с законодательством,
граждане Республики Беларусь, а также иностранные граждане и лица без гражданства,
достигшие возраста 16 лет,
согласные с положениями настоящего Устава и желающие участвовать в достижении целей объединения
\end{numberedpars}
\begin{numberedpars}
Прием в члены \OO\ осуществляется решением Правления
на основании письменного заявления вступающего,
и после уплаты им вступительного взноса.
\end{numberedpars}
\begin{numberedpars}
Размер вступительного и членских взносов, порядок их уплаты определяется Правлением \OO.
Правление \OO\ может принять решение об освобождении отдельных граждан от уплаты вступительного и членских взносов, 
для отдельных членов \OO\ размер вступительного и членских взносов может быть снижен.
\end{numberedpars}
%% В соответствии с Налоговым кодексом Республики Беларусь членские взносы не являются внереализационным доходом (не облагаются налогом на прибыль) в соответствии с размером, установленным в Уставе общественного объединения. Поэтому размер членских и вступительных взносов лучше прописывать в Уставе объединения, хотя бы «от-до»
\begin{numberedpars}
Прекращение членства может быть осуществлено путем добровольного выхода из членов 
либо в случае исключения из числа членов \OO.
\end{numberedpars}
\begin{numberedpars}
Добровольный выход из членов \OO осуществляется путем подачи письменного заявления членом \OO Правлению \OO, при этом членство считается прекращенным с даты, указанной в заявлении. Правление принимает заявление о добровольном выходе к сведению. 
\end{numberedpars}
\begin{numberedpars}
Решение об исключении из числа членов \OO\ может быть принято Правлением \OO.
Основанием для исключения из числа членов \OO\ может служить
нарушение устава членом \OO,
нанесение вреда репутации \OO,
а также нарушение членом \OO\ морально-этических норм.
Членство считается прекращенным с даты, указанной в решении Правления \OO\ об
исключении из числа членов \OO.
\end{numberedpars}

\begin{numberedpars}
Член \OO\ имеет право:
\end{numberedpars}
\begin{itemize}
\item принимать участие в организованных мероприятиях \OO, в заседаниях высшего органа;
\item избирать и выдвигать свою кандидатуру на выборах в выборные органы \OO 
по достижению совершеннолетия;
\item получать от органов и должностных лиц \OO\ информацию,
касающуюся деятельности \OO;
\item обжаловать решения органов и должностных лиц \OO\ в порядке,
предусмотренном Уставом и законодательством;
\item вносить предложения относительно деятельности \OO\
на рассмотрение выборных и высшего органов объединения;
\item вносить добровольные пожертвования для поддержки деятельности \OO;
\item свободного выхода из членов \OO.
\end{itemize}

\begin{numberedpars}
Член \OO\ обязан:
\end{numberedpars}
\begin{itemize}
\item выполнять требования Устава;
\item участвовать в работе по выполнению целей и задач \OO;
\item уплачивать членские взносы (кроме случаев, 
когда лицо было в установленном Уставом порядке освобождено от уплаты членских взносов);
\item не совершать действий, наносящих материальный ущерб или причиняющих вред
 репутации \OO.
\end{itemize}

\begin{numberedpars}
Учет членов в \OO\ осуществляется Правлением \OO путем ведения списка членов,
который редактируется по мере необходимости и обновляется по мере вступления и выбытия членов.
Правление своим решением назначает из своего состава лицо, ответственное за ведение списка членов.
\end{numberedpars}

\newpage\section{ВЫСШИЙ И ВЫБОРНЫЕ ОРГАНЫ ПОО \OONAME}

\begin{numberedpars}
\OO\ является цельным общественным объединением, не имеющим в своем
составе организационных структур.
\end{numberedpars}
\begin{numberedpars}
Высшим органом \OO\ является Общее Собрание \OO\ (далее "--- Общее Собрание),
которое созывается Правлением \OO\ по мере необходимости,
но не реже одного раза в один год.\\
Общее Собрание может быть созвано по требованию Правления \OO\ либо Ревизора
\OO, либо по инициативе не менее пяти членов \OO.\\
Время, место проведения, повестка дня Общего Собрания определяются Правлением \OO\
и сообщаются членам объединения не позднее, чем за 5 (пять) дней до дня Общего Собрания.
\end{numberedpars}
\begin{numberedpars}
Общее Собрание считается правомочным, если в нем участвует не менее 10 (десяти) членов \OO\
каждый из которых имеет на Общем Собрании один голос.
Решения Общего Собрания принимаются открытым голосованием 
простым большинством голосов от числа участвующих в нем членов \OO.
\end{numberedpars}
\begin{numberedpars}
К компетенции Общего Собрания относится:
\end{numberedpars}
\begin{itemize}
\item определение основных направлений и форм деятельности \OO 
в соответствии с уставными целями, задачами, предметом и методами деятельности;
\item утверждение названия общественного объединения;
\item принятие Устава \OO, внесение в него изменений и/или дополнений,
кроме случаев, когда в соответствии с законодательством и Уставом изменения и/или дополнения в Устав могут быть внесены Правлением;
\item избрание из числа достигших возраста 18 лет членов \OO сроком на один год Правления в составе не менее пяти человек;
Количественный состав членов Правления определяется Общим Cобранием;
\item избрание из числа членов Правления сроком на один год Председателя Правления;
\item избрание из числа членов Правления сроком на один год заместителя Председателя Правления;
\item избрание из числа достигших возраста 18 лет членов \OO сроком на один год Ревизора;
\item заслушивание и утверждение отчетов Правления, Ревизора;
\item утверждение образцов печати, штампов, символики \OO;
\item принятие решения о реорганизации или ликвидации \OO.
\end{itemize}

Общее Собрание может принять к рассмотрению любой другой вопрос деятельности \OO.

\begin{numberedpars}
В период между Общими Собраниями деятельностью \OO\ руководит
Правление, являющееся руководящим органом \OO.
Заседания Правления созываются Председателем Правления по мере необходимости, 
но не реже одного раза в четыре месяца.
Правление правомочно, если на его заседании присутствует не менее половины членов Правления.
Решения Правления принимаются открытым голосованием 
простым большинством голосов от числа участвующих в заседании членов Правления.
\end{numberedpars}
\begin{numberedpars}
Правление \OO:
\end{numberedpars}
\begin{itemize}
\item организует деятельность \OO, исисходя из его целей, задач, 
предмета и методов деятельности и в соответствии с решениями Общего Собрания;;
\item созывает и организует работу Общего Собрания;
\item утверждает отчеты Председателя Правления о доходах и расходах \OO\ 
и сметы расходов на будущий период;
\item принимает решения об участии в создании или вступлении в союзы и ассоциации;
\item принимает решения о создании и ликвидации юридических лиц, созданных при
участии \OO\ в соответствии с законодательством, утверждает их
уставы и руководителей;
\item вносит изменения и/или дополнения в Устав, связанные с переменой юридического адреса
либо обусловленные изменениями в законодательстве;
\item утверждает штатное расписание и должностные оклады, нанимает и увольняет
штатных работников;
\item принимает решения о приобретении, распоряжении и отчуждении имущества \OO;
\item решает иные вопросы деятельности \OO, не относящиеся согласно
Уставу к компетенции других органов \OO.
\end{itemize}
\begin{numberedpars}
Председатель Правления является руководителем юридического лица.
\end{numberedpars}
\begin{numberedpars}
Председатель Правления:
\end{numberedpars}
\begin{itemize}
\item созывает заседания Правления, определяет время и место их проведения,
оповещает всех членов Правления не позднее чем за два дня до заседания;
\item председательствует на заседаниях Правления;
\item осуществляет контроль за выполнением решений Правления;
\item обеспечивает выполнение решений Общего Собрания и Правления;
\item без доверенности действует от имени \OO, представляет его интересы;
\item принимает решения и издает распоряжения по текущим вопросам деятельности \OO;
\item заключает договора от имени \OO, выдает доверенности;
\item открывает расчетные и другие счета в банках;
\item распоряжается имуществом и средствами \OO\ в пределах, устанавливаемых Правлением;
\end{itemize}
При отсутствии Председателя Правления все его обязанности исполняет заместитель Председателя Правления.
\begin{numberedpars}
Для осуществления внутренней проверки финансово-хозяйственной деятельности,
а также внутреннего контроля за соответствием деятельности \OO\
учредительным документам и законодательству Общее Собрание избирает Ревизора, подотчетного Общему Собранию.
\end{numberedpars}
\begin{numberedpars}
Ревизор \OO:
\end{numberedpars}
\begin{itemize}
\item контролирует деятельность \OO;
\item проверяет бухгалтерские счета и книги, просматривает документы \OO\ в любое время
в период действия своих полномочий;
\item проверяет обоснованность ответов на письма, жалобы, предложения членов \OO;
\item проверяет организацию делопроизводства и отчетности \OO.
\end{itemize}
\begin{numberedpars}
Ревизор проводит проверки по мере необходимости, но не реже одного раза в один год.
Ревизор в случае необходимости вправе привлекать к своей работе специалистов для консультаций и участия в проведении ревизий.
\end{numberedpars}
\begin{numberedpars}
Решения коллегиальных органов \OO\ оформляются протоколами.
Председатель Правления издает приказы и распоряжения.
Ревизор оформляет проверки актами и справками.
\end{numberedpars}
\begin{numberedpars}
Решение Председателя или Правления может быть обжаловано Ревизору.
Ревизор в течение месяца обязан рассмотреть жалобу и вынеси по ней решение.
Решение Ревизора может быть обжаловано на Общем Собрании, при этом жалоба подается через Правление.
Ревизор в месячный срок обязан принять решение по существу рассматриваемой жалобы.
Ответ компетентного органа направляется в пятидневный срок заявителю.
Общее Собрание может принять к рассмотрению любой вопрос по жалобе на неправомерные действия
должностных лиц и выборных органов объединения.
Решения Общего Собрания могут быть обжалованы в суд в порядке, установленном законодательством.
\end{numberedpars}
\begin{numberedpars}
Решение о внесении изменений и/или дополнений в Устав 
принимается компетентным органом в соответствии с настоящим Уставом 
в порядке определенном для принятия иных решений соответствующим органом.
\end{numberedpars}

\newpage\section{ФИНАНСОВЫЕ СРЕДСТВА И ИМУЩЕСТВО}


\begin{numberedpars}
\OO\ может иметь в собственности любое имущество, необходимое
для материального обеспечения уставной деятельности, за исключением объектов,
которые, согласно закону, могут находиться только в собственности государства.
\end{numberedpars}
\begin{numberedpars}
Денежные средства и имущество \OO\ формируются из:
\end{numberedpars}
\begin{itemize}
\item вступительных и членских взносов;
\item добровольных пожертвований физических и юридических лиц;
\item поступлений от проводимых в уставных целях мероприятий в соответствии с законодательством;
\item отчислений от созданных \OO\ предприятий;
\item других поступлений, не запрещенных законодательством.
\end{itemize}
\begin{numberedpars}
Денежные средства \OO\ расходуются на
реализацию уставных целей и задач и не могут перераспределяться между его членами.
\end{numberedpars}

\begin{numberedpars}
\OO\ не отвечает по обязательствам своих членов, а они не отвечают по его обязательствам.
\end{numberedpars}
\begin{numberedpars}
В случае прекращения членства в \OO\ финансовые средства и имущество,
переданные его членами \OO\ в собственность безвозмездно, возврату не подлежат.
Материальные средства, переданные \OO\
его членами во временное владение и пользование, возвращаются в соответствии с
условиями договоров, на основании которых это владение и пользование осуществлялось.
\end{numberedpars}


\newpage\section{ПРЕКРАЩЕНИЕ ДЕЯТЕЛЬНОСТИ ПОО \OONAME}


\begin{numberedpars}
Прекращение деятельности \OO\ происходит путем его реорганизации или ликвидации.
\end{numberedpars}
\begin{numberedpars}
Реорганизация \OO\ производится по решению Общего Собрания.
Ликвидация \OO\ производится по решению Общего Собрания, либо по решению суда.
Решение о ликвидации направляется в регистрирующий орган и публикуется в периодическом печатном издании,
определенном актами законодательства.
Ликвидация производится ликвидационной комиссией (ликвидатором), 
созданной (назначенным) органом, принявшим решение о ликвидации.
\end{numberedpars}
\begin{numberedpars}
При ликвидации средства и имущество \OO,
оставшиеся после полного удовлетворения всех имущественных требований кредиторов,
используются на цели, предусмотренные Уставом, если денежные средства и иное имущество объединения,
в соответствии с законодательными актами, не подлежат обращению в доход государства.
\end{numberedpars}


\end{document}
