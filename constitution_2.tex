\documentclass[a4paper,fontsize=14pt,titlepage]{scrartcl}

\usepackage[T2A]{fontenc}
\usepackage[utf8]{inputenc}
\usepackage[russian]{babel}
\usepackage{indentfirst}
\usepackage{graphicx}
\usepackage[left=2cm,top=2cm,right=1.7cm,bottom=2cm,nohead,nofoot]{geometry}
\usepackage[font={small}]{caption}
\usepackage{wrapfig}
\usepackage{enumitem}

\setlist{nolistsep}
%\setlist[1]{\labelindent=\parindent}
%\setlist[itemize]{leftmargin=*}
\setlist[itemize,1]{label=---}

\setkomafont{section}{\normalfont\bfseries\centering}
\setkomafont{paragraph}{\normalfont}
\setkomafont{subparagraph}{\normalfont}
\setkomafont{title}{\rmfamily \huge}

\newcommand{\longpage}{\enlargethispage{\baselineskip}}
\newcommand{\shortpage}{\enlargethispage{-\baselineskip}}

% deep magic follows, don't touch it!
\makeatletter
\@newctr{paragraph}[section]
\@newctr{subparagraph}[paragraph]
\newenvironment{numberedpars}{%
  \addtocounter{secnumdepth}{1}
  \renewcommand\theparagraph{\arabic{section}.\arabic{paragraph}.}
  \renewcommand\@seccntformat[1]
  {\expandafter\ifx\csname##1\endcsname\paragraph\csname
  the##1\endcsname\else\csname the##1\endcsname\fi}
  \def\paragraph{\par\addvspace{3.25ex \@plus1ex \@minus.2ex}{\raggedsection\normalfont\sectfont\nobreak\size@paragraph\refstepcounter{paragraph}\@seccntformat{paragraph}}\,}
  \let\old@par=\par
  \def\new@par{\let\par=\old@par\paragraph{}\let\par=\new@par}
  \let\par=\new@par
  \par
}{
  \addtocounter{secnumdepth}{-1}
}
\newenvironment{numberedsubpars}{%
  \addtocounter{secnumdepth}{1}
  \renewcommand\thesubparagraph{\arabic{section}.\arabic{paragraph}.\arabic{subparagraph}.}
  \renewcommand\@seccntformat[1]
  {\expandafter\ifx\csname##1\endcsname\subparagraph\csname
  the##1\endcsname\else\csname the##1\endcsname\fi}
  \def\subparagraph{\par\addvspace{3.25ex \@plus1ex \@minus .2ex}{\raggedsection\normalfont\sectfont\nobreak\size@paragraph\refstepcounter{subparagraph}\@seccntformat{subparagraph}}\,}
  \let\old@@par=\par
  \def\new@@par{\let\par=\old@par\subparagraph{}\let\par=\new@@par}
  \let\par=\new@@par
  \par
}{
  \addtocounter{secnumdepth}{-1}
  \let\par=\old@@par
}

\def\emph{\textbf}

\renewcommand\thesection{\arabic{section}.}
\let\@@@section=\section
\renewcommand\section[1]{\@@@section{\MakeUppercase{#1}}}
% end of magic

\makeatother

\sloppy

\begin{document}

\setkomafont{title}{\bfseries}

\date{\large г. Минск\\2013}

\titlehead{\raggedleft \begin{minipage}{15em}%
Принят на\\%
Учредительной конференции\\%
общественного объединения\\%
«Открытая лаборатория\\%
технического творчества»\\%
25 июля 2013 года%%
\end{minipage}}


\title{\Large%
УСТАВ\\%
\textbf{\large ОБЩЕСТВЕННОГО ОБЪЕДИНЕНИЯ}\\%
«Открытая лаборатория технического творчества»\\[1ex]%
(ОО «Открытая лаборатория\\%
технического творчества»)\\[4ex]%
СТАТУТ\\%
\textbf{\large ГРАМАДСКАГА АБ’ЯДНАННЯ}\\%
«Адкрытая лабараторыя тэхнічнай творчасці»\\[1ex]%
(ГА «Адкрытая лабараторыя\\%
тэхнічнай творчасці»)%
}
\maketitle


\section{ОБЩИЕ ПОЛОЖЕНИЯ}


\begin{numberedpars}
Общественное объединение «Открытая лаборатория технического творчества» (далее по тексту "--- ОО «Открытая лаборатория технического творчества») является добровольным объединением граждан, объединившихся на основе общности интересов в установленном законодательством порядке.
\end{numberedpars}
\begin{numberedpars}
Полное название объединения:
на русском языке: \emph{Общественное объединение «Открытая лаборатория технического творчества»};
на белорусском языке: \emph{Грамадскае аб’яднанне «Адкрытая лабараторыя тэхнічнай творчасці»}.
Сокращенное название объединения:
на русском языке: \emph{ОО «Открытая лаборатория технического творчества»};
на белорусском языке: \emph{ГА «Адкрытая лабараторыя тэхнічнай творчасці»}.
\end{numberedpars}
\begin{numberedpars}
ОО «Открытая лаборатория технического творчества»
имеет статус местного общественного объединения.
Территория распространения деятельности ОО «Открытая лаборатория технического творчества» "--- г. Минск.
\end{numberedpars}
\begin{numberedpars}
ОО «Открытая лаборатория технического творчества»
создается и действует в соответствии с законодательством Республики Беларусь
и настоящим Уставом.
\end{numberedpars}
\begin{numberedpars}
ОО «Открытая лаборатория технического творчества»
с момента регистрации является юридическим лицом,
несет самостоятельную ответственность по своим обязательствам,
имеет обособленное имущество,
самостоятельный баланс и счета в банках,
от своего имени выступает во взаимоотношениях с юридическими и физическими лицами,
может быть истцоми ответчиком в судах,
имеет печать,
бланки со своим наименованием,
может иметь собственную символику,
которая подлежит регистрации в установленном порядке.
\end{numberedpars}
\begin{numberedpars}
ОО «Открытая лаборатория технического творчества»
может вступать в международные общественные объединения,
созданные на территории иностранных государств,
союзы, участвовать в их создании,
поддерживать прямые международные контакты и связи в соответствии с законодательством.
\end{numberedpars}
\begin{numberedpars}
Юридический адрес ОО «Открытая лаборатория технического творчества»:
\end{numberedpars}



\newpage\section{ЦЕЛЬ, ЗАДАЧИ, ПРЕДМЕТ И МЕТОДЫ ДЕЯТЕЛЬНОСТИ}


\begin{numberedpars}
Целью ОО «Открытая лаборатория технического творчества» является.
\end{numberedpars}
\begin{numberedpars}
Задачами ОО «Открытая лаборатория технического творчества» являются:
\end{numberedpars}
\begin{numberedpars}
Предметом деятельности ОО «Открытая лаборатория технического творчества» является
\end{numberedpars}
\begin{numberedpars}
Для достижения своей цели и решения задач ОО «Открытая лаборатория технического творчества» в порядке, установленном законодательством, использует следующие методы деятельности:
Деятельность, на осуществление которой требуется специальное разрешение (лицензия), осуществляется только после получения необходимого разрешения (лицензии) в установленном порядке.
\end{numberedpars}



\newpage\section{ЧЛЕНСТВО, ПРАВА И ОБЯЗАННОСТИ ЧЛЕНОВ ОО «Открытая лаборатория технического творчества»}


\begin{numberedpars}
Членами ОО «Открытая лаборатория технического творчества» могут быть, в соответствии с законодательством,
граждане Республики Беларусь, а также иностранные граждане и лица без гражданства, достигшие возраста 18 лет, признающие Устав и желающие участвовать в
достижении его цели.
Лица, не достигшие 16 лет, могут вступать в ОО «Открытая лаборатория технического творчества» с письменного
согласия своих законных представителей.
\end{numberedpars}
\begin{numberedpars}
Прием в члены ОО «Открытая лаборатория технического творчества» осуществляется решением (какого органа) по заявлению вступающего. Прекращение членства может быть осуществлено
путем выхода из членов объединения либо в случае исключения из числа членов
ОО «Открытая лаборатория технического творчества».
\end{numberedpars}
\begin{numberedpars}
Размер вступительного и членских взносов, порядок их уплаты определяется (каким органом). (Либо в Уставе конкретно определяется размер вступительного и членских взносов). Орган может принять решение об освобождении отдельных граждан от уплаты вступительного и членских взносов, для отдельных членов
ОО «Открытая лаборатория технического творчества» размер вступительного и членских взносов может быть снижен (в
случае наличия членских взносов).
\end{numberedpars}
\begin{numberedpars}
Выход из членов ОО «Открытая лаборатория технического творчества» осуществляется путем подачи заявления в (какой орган), при этом членство считается прекращенным с даты, указанной в заявлении.
\end{numberedpars}
\begin{numberedpars}
Решение об исключении из числа членов ОО «Открытая лаборатория технического творчества» может быть принято (каким органом) в случае грубого нарушения Устава членом объединения. Членство считается прекращенным с даты, указанной в решении (какого органа) об
исключении из числа членов ОО «Открытая лаборатория технического творчества».
\end{numberedpars}

\begin{numberedpars}
Член ОО «Открытая лаборатория технического творчества» имеет право:
\end{numberedpars}
\begin{itemize}
\item принимать участие в мероприятиях ОО «Открытая лаборатория технического творчества», в заседаниях высшего органа;
\item избирать и выдвигать свою кандидатуру на выборах в выборные органы
ОО «Открытая лаборатория технического творчества»
(в случае, если членство с 16 и менее лет, то определяется: по
достижении совершеннолетия);
\item получать от органов и должностных лиц ОО «Открытая лаборатория технического творчества» информацию, касающуюся деятельности ОО «Открытая лаборатория технического творчества»;
\item обжаловать решения органов и должностных лиц ОО «Открытая лаборатория технического творчества» в порядке,
предусмотренном Уставом и законодательством;
\item вносить предложения относительно деятельности ОО «Открытая лаборатория технического творчества» на рассмотрение выборных органов объединения;
\item вносить добровольные пожертвования для поддержки деятельности ОО «Открытая лаборатория технического творчества»;
\item свободного выхода из членов ОО «Открытая лаборатория технического творчества».
\end{itemize}

\begin{numberedpars}
Член ОО «Открытая лаборатория технического творчества» обязан:
\end{numberedpars}
\begin{itemize}
\item выполнять требования Устава;
\item участвовать в работе по выполнению цели и задач ОО «Открытая лаборатория технического творчества»;
\item не совершать действий, наносящих материальный ущерб или причиняющих вред
деловой репутации ОО «Открытая лаборатория технического творчества».
\end{itemize}
\begin{numberedpars}
Учет членов в ОО «Открытая лаборатория технического творчества» осуществляется  (каким органом) путем
ведения списка членов, который редактируется по мере необходимости и обновляется по мере вступления и выбытия членов.
\end{numberedpars}


\newpage\section{СТРУКТУРА ОО «Открытая лаборатория технического творчества»}


\begin{numberedpars}
Здесь может быть несколько вариантов:
1. ОО «Открытая лаборатория технического творчества» является цельным общественным объединением, без территориальных организационных структур (кроме международных общественных объединений).
2. Либо расписывается структура общественного объединения: порядок создания и
прекращения деятельности организационных структур, название, состав, порядок
избрания, порядок и периодичность созыва, сроки полномочий органов организационных структур и их компетенция; порядок принятия и обжалования решений органов
организационных структур общественного объединения; пределы распоряжения
имуществом общественного объединения его организационными структурами.
Например, для международного общественного объединения:
\end{numberedpars}
\begin{numberedpars}
Основу ОО «Открытая лаборатория технического творчества» составляют отделения в Республике Беларусь и
 (указать, в какой стране), создаваемые на Учредительном собрании.
В составе белорусского отделения должно быть не менее десяти членов. По решению  (какого органа) отделения
ОО «Открытая лаборатория технического творчества» могут быть созданы в других странах при наличии трех и более
членов ОО «Открытая лаборатория технического творчества», граждан той страны, где создается отделение.
\end{numberedpars}
\begin{numberedpars}
Все отделения равны между собой.
\end{numberedpars}
\begin{numberedpars}
Высшим органом отделения ОО «Открытая лаборатория технического творчества» является Общее собрание (варианты: Конференция, Собрание), которое созывается по мере необходимости,
но не реже одного раза в  года. Общее собрание определяет формы работы,
компетенцию, основные направления деятельности отделений, а также избирает
Председателя и Ревизора отделения (могут быть избраны коллегиальный руководящий и контрольно-ревизионный органы, а также исполнительный орган). Общее
собрание принимает решения, обязательные для органов отделения. Общее собрание считается правомочным при наличии более половины членов ОО «Открытая лаборатория технического творчества»,
проживающих на соответствующей территории. Решения принимаются простым
большинством голосов от числа присутствующих. Общее собрание проводится по
решению  (какого органа) либо по требованию не менее  членов ОО «Открытая лаборатория технического творчества», проживающих на соответствующей территории.
\end{numberedpars}
\begin{numberedpars}
Председатель отделения избирается Общим собранием сроком на  года.
Председатель отделения осуществляет руководство деятельностью отделения в
период между созывами Общего собрания, а также председательствует на Общем
собрании.
\end{numberedpars}
\begin{numberedpars}
Внутреннюю проверку финансово-хозяйственной деятельности отделения, а
также внутренний контроль за соответствием деятельности отделения законодательству и Уставу осуществляет Ревизор отделения, избираемый Общим собранием сроком на  года. Ревизор отделения проводит проверки по мере необходимости, но не реже .
\end{numberedpars}
\begin{numberedpars}
Решения Председателя и Ревизора отделений обжалуются в порядке, предусмотренном настоящим Уставом.
\end{numberedpars}
\begin{numberedpars}
Отделения имеют те же цели и задачи, что и ОО «Открытая лаборатория технического творчества» и действуют в
соответствии с Уставом.
\end{numberedpars}
\begin{numberedpars}
По решению  (какого органа) отделения ОО «Открытая лаборатория технического творчества» могут наделяться
правами юридического лица.
\end{numberedpars}
\begin{numberedpars}
Отделения ОО «Открытая лаборатория технического творчества» ликвидируются в порядке, установленном законодательством. Отделения ОО «Открытая лаборатория технического творчества» могут быть ликвидированы по решению
 (какого органа).
\end{numberedpars}
\begin{numberedpars}
Отделения ОО «Открытая лаборатория технического творчества» не имеют права распоряжаться имуществом
ОО «Открытая лаборатория технического творчества».
\end{numberedpars}



\newpage\section{ВЫСШИЙ И ВЫБОРНЫЕ ОРГАНЫ ОО «Открытая лаборатория технического творчества»}


\begin{numberedpars}
Высшим органом ОО «Открытая лаборатория технического творчества» является Общее Собрание, которая созывается  (каким органом) по
мере необходимости, но не реже одного раза в  года. Конференция может быть
созвана по требованию  (каких органов), либо по инициативе не менее
членов ОО «Открытая лаборатория технического творчества». Время, место проведения, повестка дня Конференции определяются  (каким органом) и сообщаются членам объединения не позднее,
чем за  дней до дня Конференции. Конференция считается правомочной, если в
ней участвует не менее  членов ОО «Открытая лаборатория технического творчества». Форма и порядок голосования устанавливается Конференцией.
\end{numberedpars}
\begin{numberedpars}
К компетенции Конференции относится (например):
\end{numberedpars}
\begin{itemize}
\item определение основных направлений и форм деятельности ОО «Открытая лаборатория технического творчества»;
утверждение названия, принятие Устава ОО «Открытая лаборатория технического творчества», внесение в него изменений и/или дополнений (кроме случаев, когда в соответствии с законодательством изменения и/или дополнения в Устав могут быть внесены Правлением)
(обязательный пункт);
\item избрание сроком на  года Правления в составе не менее -х человек,
Председателя Правления, Заместителя Председателя Правления, Ревизионной
комиссии в составе не менее -х человек, Председателя Ревизионной комиссии
(либо Председатель Правления, заместитель Председателя Правления, Председатель Ревизионной комиссии избираются на заседаниях соответствующих
органов, что прописывается в Уставе) (обязательный пункт) (количественный
состав членов Правления и Ревизионной комиссии может быть точно определен
в Уставе);
\item заслушивание отчетов Правления, Ревизионной комиссии;
\item принятие решения о реорганизации или ликвидации ОО «Открытая лаборатория технического творчества» (обязательный пункт).
\end{itemize}
Конференция может принять к рассмотрению любой другой вопрос деятельности
ОО «Открытая лаборатория технического творчества».
\begin{numberedpars}
В период между Конференциями деятельностью ОО «Открытая лаборатория технического творчества» руководит
Правление, являющееся руководящим органом ОО «Открытая лаборатория технического творчества». Правление избирается сроком на  года. Заседания Правления созываются Председателем Правления по мере необходимости, но не реже .
\end{numberedpars}
\begin{numberedpars}
Правление правомочно, если на его заседании присутствует не менее  членов Правления.
\end{numberedpars}
\begin{numberedpars}
Правление ОО «Открытая лаборатория технического творчества» (например):
\end{numberedpars}
\begin{itemize}
\item организует деятельность ОО «Открытая лаборатория технического творчества», исходя из его цели, задач и методов;
\item созывает и организует работу Конференции;
\item утверждает отчеты о доходах и расходах ОО «Открытая лаборатория технического творчества» и сметы расходов на
будущий период;
\item принимает решения об участии в создании или вступлении в союзы и ассоциации;
\item принимает решения о создании и ликвидации юридических лиц, созданных при
участии ОО «Открытая лаборатория технического творчества» в соответствии с законодательством, утверждает их
уставы и руководителей (в случае необходимости);
\item вносит изменения и/или дополнения в Устав, связанные с переменой юридического адреса либо обусловленные изменениями в законодательстве (обязательный
пункт);
\item утверждает штатное расписание и должностные оклады, нанимает и увольняет
штатных работников;
\item принимает решения о приобретении, распоряжении и отчуждении имущества
ОО «Открытая лаборатория технического творчества» (обязательный пункт) (либо необходимо наделить компетенцией принимать решения о приобретении имущества и распоряжении им иной
орган);
\item заслушивает отчеты Председателя Правления, Заместителя Председателя Правления, руководителей созданных ОО «Открытая лаборатория технического творчества» юридических лиц;
\item утверждает образцы печати, штампов, символики ОО «Открытая лаборатория технического творчества»;
\item решает иные вопросы деятельности ОО «Открытая лаборатория технического творчества», не относящиеся согласно
Уставу к компетенции других органов ОО «Открытая лаборатория технического творчества».
\end{itemize}
\begin{numberedpars}
Председатель Правления является руководителем юридического лица. Председатель Правления входит в состав Правления по должности.
\end{numberedpars}
\begin{numberedpars}
Председатель Правления (например):
\end{numberedpars}
\begin{itemize}
\item созывает заседания Правления, определяет время и место их проведения, оповещает всех членов Правления не позднее чем за  дней до заседания;
\item председательствует на заседаниях Правления;
\item осуществляет контроль за выполнением решений Правления;
\item обеспечивает выполнение решений Конференции, Правления;
без доверенности действует от имени ОО «Открытая лаборатория технического творчества», представляет его интересы;
принимает решения по вопросам, которые не отнесены к компетенции Конференции, Правления;
принимает решения и издает распоряжения по текущим вопросам деятельности
\item ОО «Открытая лаборатория технического творчества»;
\item заключает договоры от имени ОО «Открытая лаборатория технического творчества», выдает доверенности;
\item открывает расчетные и другие счета в банках (обязательный пункт);
\item распоряжается имуществом и средствами ОО «Открытая лаборатория технического творчества» в пределах, устанавливаемых Правлением;
\item решает другие вопросы, связанные с деятельностью ОО «Открытая лаборатория технического творчества» и не отнесенные к компетенции иных органов.
\end{itemize}
При отсутствии Председателя Правления все его обязанности исполняет Заместитель Председателя Правления.
\begin{numberedpars}
Для осуществления внутренней проверки финансово-хозяйственной деятельности, а также внутреннего контроля за соответствием деятельности ОО «Открытая лаборатория технического творчества»
учредительным документам и законодательству Конференция избирает Ревизионную
комиссию (варианты: Контрольно-Ревизионная комиссия) (в местной организации
может быть только Ревизор, в республиканских и
орган), в составе не менее  членов, подотчетную Конференции.
\end{numberedpars}
\begin{numberedpars}
Ревизионную комиссию возглавляет Председатель Ревизионной комиссии, избираемый Конференцией на срок ее полномочий. Председатель Ревизионной комиссии
созывает заседания, руководит работой Ревизионной комиссии, председательствует
на ее заседаниях.
\end{numberedpars}
\begin{numberedpars}
Ревизионная комиссия ОО «Открытая лаборатория технического творчества» (например):
\end{numberedpars}
\begin{itemize}
\item контролирует деятельность ОО «Открытая лаборатория технического творчества»;
\item проверяет бухгалтерские счета и книги, просматривает документы ОО «Открытая лаборатория технического творчества» в любое время в период действия своих полномочий;
\item проверяет обоснованность ответов на письма, жалобы, предложения членов ОО
«Открытая лаборатория технического творчества»;
\item проверяет организацию делопроизводства и отчетности ОО «Открытая лаборатория технического творчества».
\end{itemize}
\begin{numberedpars}
Члены Ревизионной комиссии не могут быть избраны в Правление ОО «Открытая лаборатория технического творчества», однако могут принимать участие в его работе с правом совещательного голоса.
\end{numberedpars}
\begin{numberedpars}
Ревизионная комиссия проводит проверки по мере необходимости, но не реже. Ревизионная комиссия в случае необходимости вправе привлекать к
своей работе специалистов для консультаций и участия в проведении ревизий. Заседания Ревизионной комиссии проводятся не реже  и правомочны в случае
присутствия на заседании не менее  членов Ревизионной комиссии (в случае,
если в объединении только Ревизор, это не указывается)
\end{numberedpars}
\begin{numberedpars}
Решения коллегиальных органов ОО «Открытая лаборатория технического творчества» оформляются протоколами. Председатель Правления издает приказы и распоряжения. Ревизионная комиссия
оформляет проверки актами и справками.
\end{numberedpars}
\begin{numberedpars}
Решения коллегиальных органов ОО «Открытая лаборатория технического творчества» (и его организационных
структур) принимаются открытым голосованием простым большинством голосов
от числа присутствующих, если Уставом не оговорено иное.
\end{numberedpars}
\begin{numberedpars}
Решение Председателя отделения может быть обжаловано Ревизору. Ревизор
в течение месяца обязан рассмотреть жалобу и вынеси по ней решение. Решение Ревизора может быть обжаловано на Общем Собрании отделения (для международного общественного объединения. Решения Общего собрания отделения, Правления,
Председателя Правления могут быть обжалованы в Ревизионной комиссии. Решения
Ревизионной комиссии обжалуются в Конференции (может быть предусмотрен
иной порядок обжалования решений выборных органов). Ревизионная комиссия в месячный срок обязана принять решение по существу рассматриваемой жалобы. Ответ компетентного органа направляется в пятидневный срок заявителю. Конференция
может принять к рассмотрению любой вопрос по жалобе на неправомерные действия
должностных лиц и выборных органов объединения.
\end{numberedpars}


\newpage\section{Финансовые средства и имущество}


\begin{numberedpars}
ОО «Открытая лаборатория технического творчества» может иметь в собственности любое имущество, необходимое
для материального обеспечения уставной деятельности, за исключением объектов,
которые, согласно закону, могут находиться только в собственности государства.
\end{numberedpars}
\begin{numberedpars}
Денежные средства и имущество ОО «Открытая лаборатория технического творчества» формируются из:
\end{numberedpars}
\begin{itemize}
\item вступительных и членских взносов (в случае их наличия);
\item добровольных пожертвований физических и юридических лиц;
\item поступлений от проводимых в уставных целях мероприятий в соответствии с законодательством;
\item отчислений от созданных ОО «Открытая лаборатория технического творчества» предприятий;
\item других поступлений, не запрещенных законодательством.
\end{itemize}
Средства ОО «Открытая лаборатория технического творчества» расходуются на реализацию уставных цели и задач и
не могут перераспределяться между его членами.
\begin{numberedpars}
ОО «Открытая лаборатория технического творчества» несет ответственность по принятым на себя обязательствам
всем принадлежащим ему имуществом. ОО «Открытая лаборатория технического творчества» не отвечает по обязательствам своих членов, а они не отвечают по его обязательствам.
\end{numberedpars}
\begin{numberedpars}
В случае прекращения членства в ОО «Открытая лаборатория технического творчества» финансовые средства и
имущество, переданные его членами ОО «Открытая лаборатория технического творчества» в собственность безвозмездно, возврату не подлежат. Материальные средства, переданные ОО «Открытая лаборатория технического творчества»
его членами во временное владение и пользование, возвращаются в соответствии с
условиями договоров, на основании которых это владение и пользование осуществлялось.
\end{numberedpars}


\newpage\section{ПРАВА ОО «Открытая лаборатория технического творчества»}


\begin{numberedpars}
Для достижения уставных цели и задач ОО «Открытая лаборатория технического творчества» вправе:
\end{numberedpars}
\begin{itemize}
\item осуществлять деятельность, направленную на достижение уставной цели;
\item беспрепятственно получать и распространять информацию, имеющую отношение к своей деятельности, в порядке, установленном законодательством, учреждать собственные и пользоваться в установленном порядке государственными
средствами массовой информации, осуществлять издательскую деятельность;
\item представлять и защищать права и законные интересы своих членов в государственных органах и иных организациях;
\item поддерживать связи с другими некоммерческими организациями;
\item создавать некоммерческие организации, вступать в союзы (ассоциации);
\item иметь иные права, предусмотренные законодательными актами.
\end{itemize}
ОО «Открытая лаборатория технического творчества» может осуществлять в установленном порядке предпринимательскую деятельность лишь постольку, поскольку она необходима для уставных целей, ради которых оно создано, соответствует этим целям и отвечает предмету деятельности. Такая деятельность может осуществляться только посредством образования коммерческих организаций и (или) участия в них.



\newpage\section{ПРЕКРАЩЕНИЕ ДЕЯТЕЛЬНОСТИ ОО «Открытая лаборатория технического творчества»}


\begin{numberedpars}
Прекращение деятельности ОО «Открытая лаборатория технического творчества» происходит путем его реорганизации или ликвидации.
\end{numberedpars}
\begin{numberedpars}
Реорганизация ОО «Открытая лаборатория технического творчества» производится по решению Конференции, если
за это проголосовало не менее  членов (в случае, если принятие данного решения
происходит не на общих основаниях). Ликвидация ОО «Открытая лаборатория технического творчества» производится
по решению Конференции, если за это проголосовало не менее  членов (в случае,
если принятие данного решения происходит не на общих основаниях), либо по решению суда. Решение о ликвидации направляется в регистрирующий орган и публикуется в периодическом печатном издании, определенном актами законодательства.
Ликвидация производится ликвидационной комиссией, созданной органом, принявшим решение о ликвидации.
\end{numberedpars}
\begin{numberedpars}
При ликвидации средства и имущество ОО «Открытая лаборатория технического творчества», оставшиеся после
полного удовлетворения всех имущественных требований кредиторов, используются на цели, предусмотренные Уставом, если денежные средства и иное имущество
объединения, в соответствии с законодательными актами, не подлежат обращению
в доход государства.
\end{numberedpars}
\begin{numberedpars}
Ликвидация ОО «Открытая лаборатория технического творчества» влечет за собой ликвидацию его отделений (в
случае наличия отделений).
\end{numberedpars}


\end{document}
